\documentclass{article}
\usepackage[utf8]{inputenc}
\usepackage{amsfonts}
\usepackage{mathtools}


\title{Beispiel konvergierende Funktionenfolge}

\begin{document}
\maketitle

Beispiel:
\vspace{5mm}

D = [0,1];   \begin{math}
f_n(x) = x^n \quad (n \in \mathbb{N}).\end{math} Es gilt: \vspace{5mm}

\begin{math} f(x) := 
\begin{cases} 
0, & 0 \leq x < 1\\
1, & x = 1
\end{cases} \end{math}
\vspace{5mm}

$f_n$ konvergiert auf [0,1] punktweise gegen $f$.

\vspace{10mm}
\textbf{Bemerkung:} Punktweise Konvergenz von $(f_n)$ auf \textit{D} gegen $f$ bedeutet:

\vspace{5mm}
\begin{math}
\forall x \in D \quad \forall \varepsilon > 0 \quad \exists n_0 = n_0(\varepsilon, x) \in \mathbb{N} \quad \forall n \geq n_0: |f_n(x) - f(x)| < \varepsilon.
\end{math}

\end{document}
